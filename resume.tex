% LaTeX resume using res.cls
\documentclass[line,margin]{res}
\usepackage{helvetica} % uses helvetica postscript font (download helvetica.sty)

\newcommand{\tab}{\hspace*{2em}}

\begin{document}
\name{Benjamin K. Carriel}
\address{212 Linden Ave, Apt. 3, Ithaca, NY 14853}
\address{{\bf Phone:} (845) 664-5697 {\bf Email:} bkc39@cornell.edu}

\begin{resume}
  \section{EXPERIENCE}
  {\bf Apple}, {\textit{Software Engineer Intern}} \\
  I built an application that used machine learning and big data
  techniques to improve the paging system in Mac OS X.
  \vspace{-1em}

  {\bf Goldman Sachs}, {\textit{Summer Analyst}} \\
  I worked in Goldman Sachs Electronic Trading (GSET) and built an
  application that would help process high-volume trades for clients.
  \vspace{-1em}

  {\bf Cornell Daily Sun}, {\textit{Lead iOS Developer}} \\
  I started the team that builds the mobile-app for the Cornell Daily
  Sun. I worked primarily on the iOS version of the app.
  \vspace{-1em}

  {\bf Cornell Dept. of Computer Science}, \textit{Teaching Assistant,
    CS 3110} \\
  I preach the good word of functional programming to 20-40 students
  twice a week. Other responsibilities include office hours, making
  problem sets, and exam problems. The course is {\it CS 3110 : Functional
  Programming and Data Structures}
  \vspace{-1em}

  {\bf Cornell Math Support Center}, \textit{Tutor} \\
  I tutor students at all levels of Math background. Subjects range
  from pre-calculus to Analysis, Algebra, and Topology.

  \section{EDUCATION}
  {\bf Cornell University}, College of Arts and Sciences \hfill Ithaca, NY\\
  {\bf BA Mathematics} \hfill  May 2014 \\
  {\bf BA Computer Science} \hfill  May 2014 \\
  \vspace{-2em}
  \begin{center}
    {\bf SELECTED COURSEWORK}
    \begin{ncolumn}{2}
      \tab \tab {\bf Mathematics}    & \tab \tab {\bf Computer Science} \\
      Honors Analysis I-II           & Object-Oriented  Programming \\
      Honors Algebra I-II            & Functional Programming \\
      Topology                       & Systems Programming and Organization \\
      Combinatorics                  & Operating Systems \\
      Real Analysis*                 & Introduction to Algorithms \\
      Complex Analysis*              & Compilers \\
      Partial Differential Equations*& Design and Analysis of Algorithms* \\
      Differentiable Manifolds*      & The Structure of Information Networks* \\
      Algebraic Topology I-II*       & Advanced Programming Languages* \\
    \end{ncolumn}
  \end{center}
  \vspace{-1em}
  {\footnotesize{\textbf{Note:} A * indicates a course taken at the
      graduate level.}}

  \section{PROJECTS}
  \begin{tabular}{p{1.5in}l}
    {\bf ocaml-monadic} & A Haskell-style monad library in OCaml featuring \\
    & implementations of commonly used monads \\
    & and monad transformers. \\
    {\bf ocaml-data-structures} & Implementations of common data
    structures in OCaml. \\
    {\bf tex-swag} & A collection of \LaTeX\ styles and macros for
    typesetting \\
    & Mathematics and Computer Science problem sets. \\
    {\bf Project Euler} & A series of math-related programming
    challenges. \\
    & Currently over 100 problems solved.
  \end{tabular}

  \section{SKILLS}
  {\bf Programming:} OCaml, Haskell, Scala, Python, Java, C, \LaTeX \\
  {\bf Platforms:} UNIX, Git, Subversion, Spring,
  Apache TomCat, Phabricator \\
  {\bf General:} Guitar, Doumbek, Sailing, Soccer, YoYo \\
\end{resume}
\end{document}